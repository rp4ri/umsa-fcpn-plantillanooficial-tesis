\makeatletter
% TÍTULO, AUTOR Y AÑO DE PUBLICACIÓN
% En esta parte se puede editar el texto entre llaves.
\title{Título espectacular para tu tesis}\let\Title\@title
\author{Rodrigo Pari}        \let\Author\@author
\date{2022}                         \let\Date\@date

% DEFINIENDO NUEVOS MACROS
% En esta parte se definen los macros de los comandos que usaré para la portada, no se debe editar.
\newcommand{\@universidad}{}
\newcommand{\universidad}[1]{\renewcommand{\@universidad}{#1}}

\newcommand{\@facultad}{}
\newcommand{\facultad}[1]{\renewcommand{\@facultad}{#1}}

\newcommand{\@carrera}{}
\newcommand{\carrera}[1]{\renewcommand{\@carrera}{#1}}

\newcommand{\@tutor}{}
\newcommand{\tutor}[1]{\renewcommand{\@tutor}{#1}}

\newcommand{\@departamento}{}
\newcommand{\departamento}[1]{\renewcommand{\@departamento}{#1}}

\newcommand{\@pais}{}
\newcommand{\pais}[1]{\renewcommand{\@pais}{#1}}

% INFORMACIÓN ADICIONAL
% En esta parte sí se puede editar el texto entre llaves.

% Universidad
\universidad{UNIVERSIDAD MAYOR DE SAN ANDRÉS}\let\Universidad\@universidad
% Facultad
\facultad{FACULTAD DE CIENCIAS PURAS Y NATURALES}\let\Facultad\@facultad
% Carrera
\carrera{CARRERA DE MATEMÁTICAS}\let\Carrera\@carrera
% Tutor
\tutor{Dr. Latex}\let\Tutor\@tutor
% Departamento
\departamento{La Paz}\let\Departamento\@departamento
% País
\pais{Bolivia}\let\Pais\@pais

\makeatother